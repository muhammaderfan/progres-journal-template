\documentclass{iaesarticle-progres}

%%required package. add for your convenient, but do not remove the initial
\setlength{\headsep}{0.15in}
\usepackage{amsmath, amsfonts, amssymb, float, fancyhdr}
\usepackage[figuresright]{rotating}
\usepackage{authblk, graphicx, indentfirst, lastpage, lipsum}
\usepackage{pifont}
\renewcommand{\Authsep}{, }
\renewcommand{\Authand}{, }
\renewcommand{\Authands}{, }
\setlength{\affilsep}{0cm}
\renewcommand\Authfont{\normalsize}
\renewcommand\Affilfont{\normalfont\small}
\makeatletter
\renewcommand{\@biblabel}[1]{[#1]\hfill}
\makeatother
\usepackage{subfig, caption, epstopdf}
\usepackage[left=3cm, right=2.5cm, top=2.5cm, bottom=2.5cm, includehead, includefoot]{geometry}
\usepackage[justification=centering]{caption}
\captionsetup{labelsep=period}
\usepackage{titlesec}
\usepackage{xcolor}
\definecolor{shadecolor}{RGB}{91,155,213}
\titleformat{\section}
{\normalfont\normalsize\bfseries\uppercase}{\thesection}{1.7em}{}
\titleformat{\subsection}
{\normalfont\normalsize\bfseries}{\thesubsection}{.95em}{}
\titleformat{\subsubsection}
{\bfseries}{\thesubsubsection}{.2em}{}
\titlespacing*{\section}{0cm}{0.7cm}{0cm}
\usepackage{enumitem}
%\usepackage[numbers,compress]{natbib}
%\usepackage{csquotes}
\usepackage[style=apa, backend=biber]{biblatex}
\addbibresource{contoh bibfile.bib}
\usepackage[colorlinks]{hyperref}
\hypersetup{
  colorlinks   = true, 
  urlcolor     = blue,
  linkcolor    = blue, 
  citecolor   = blue 
}

%%leave copyright info to the editor
\CopyrightLine[]{}{\textit{This is an open access article under the \textcolor{blue}{\underline{CC BY-SA}} license.}
\vspace{.5em}}

%%author
\author[1]{\bfseries Penulis 1}
\author[2]{\bfseries Penulis 2}
\author[3]{\bfseries Penulis 3}
\author[4]{\bfseries Penulis 4}
\author[5]{\bfseries Penulis 5}

%%author's affiliation
\affil[1,2,3]{Program Studi Pendidikan Guru Sekolah Dasar, Universitas Mataram, Indonesia}
\affil[4,5]{Program Studi Pendidikan Fisika, Universitas Samawa, Indonesia}

%%title and shortitle (for footer)
\title{JUDUL MEMBERI GAMBARAN PENELITIAN YANG TELAH DILAKUKAN, MAKSIMUM 16 KATA}
\shorttitle{Judul Artikel Versi Singkat (Nama Penulis 1)}

%%starting
\begin{document}
\setcounter{page}{1}

%%indentation. do not change
\setlength{\parindent}{1.27cm}

%%header and footer setting. do not change
\pagestyle{fancy}
\fancyhfoffset{0cm}

%%journal info
\journalname{
\noindent\colorbox{shadecolor}
{\parbox{\dimexpr\textwidth-2\fboxsep\relax}{\textsc{PROGRES PENDIDIKAN}}}
}
\journalshortname{PROGRES PENDIDIKAN}
\journalhomepage{http://prospek.unram.ac.id/index.php/PROSPEK}
\vol{x}
\no{x}
\months{January}
\years{202x}
\pissn{2721-3374}
\eissn{2721-9348}
\DOI{10.29303}
\pagefirst{xx}
\pagelast{xx}
\IDpaper{xx}



%%build title
\maketitle

%%border setting. do not change
\hrule
\vspace{.1em}
\hrule
\vspace{.5em}
\noindent
\parbox[t][][s]{0.275\textwidth}{%
\textbf{Informasi Artikel}
\vspace{.5em}
\hrule
\vspace{.5em}
\begin{history}
\vspace{.5em}

%%article info. editor's privilege
Diserahkan Jun 9, 2018

Direvisi Nov 20, 2018

Dipublikasikan Dec 11, 2018

\vspace{.7em}
\end{history}
\vspace{.5em}
\hrule
\vspace{.5em}
\begin{keyword} 
\vspace{.5em}
%%write keyword here. separate by \sep
kata kunci pertama \sep kata kunci kedua \sep kata kunci ketiga \sep kata kunci keempat \sep kata kunci kelima
\vspace{.5em}
\end{keyword}
\vspace{\fill}
}
\parbox{0.025\textwidth}{\hspace{0.5em}}
\parbox[t][][s]{0.7\textwidth}{%
\begin{abstract}
\vspace{.3em}
%% Text of abstract
Bagian abstrak berisi ringkasan penelitian yang telah anda lakukan, bersifat informatif, memuat permasalahan apa yang ingin di pecahkan dan pendekatan apa yang digunakan serta temuan apa yang didapakan. Jumlah kata maksimum pada abstrak adalah 200 kata. Maksimum jumlah halaman artikel adalah 15 halaman. Aturan sitasi menggunakan APA edisi ketujuh.
\end{abstract}
}
\parbox[l]{\textwidth}{%
\rule{0.275\textwidth}{0.5pt} \hspace{0.5cm} \hrulefill
\\
\emph{\textbf{Penulis Korespondensi:}}
\vspace{.5em}\\
%% correspondence info. separate by \\
Nama Penulis Korespondensi,\\
Program Studi, Fakultas,\\
Universitas,\\
Alamat Universitas, Kota, Negara.\\
Email: koresponden@universitas.ac.id
}
\vspace{.5em}
\hrule
\vspace{.1em}
\hrule


%% main text

\section{PENDAHULUAN}
\label{pendahuluan}
Pendahuluan menguraikan latar belakang permasalahan yang diselesaikan, isu-isu yang terkait dengan masalah yg diselesaikan, ulasan penelitan yang pernah dilakukan sebelumnya oleh peneliti lain yg relevan dengan penelitian yang dilakukan \parencite{Maulyda2020,Erfan2020}. Bagian pendahuluan terutama berisi: (1) permasalahan penelitian; (2) rumusan tujuan penelitian; (3) acuan Pustakayang berkaitan dengan masalah yang diteliti. Panjang bagian pendahuluan sekitar 2-3 halaman \parencite{Ermiana2020}. Maksimum halaman yang diperbolehkan adalah 15 halaman.

\section{METODE PENELITIAN}
\label{metode}
Pada dasarnya bagian ini menjelaskan bagaimana penelitian itu dilakukan. Materi pokok bagian ini adalah: (1) pendekatan/ jenis penelitian; (2) sumber data; (3) teknik dan instrumen pengumpulan data; dan (4) teknik analisis data \parencite{Affandi2020}. Untuk penelitian yang menggunakan alat dan bahan, perlu dituliskan spesifikasi alat dan bahannya. Spesifikasi alat menggambarkan kecanggihan alat yang digunakan sedangkan spesifikasi bahan menggambarkan macam bahan yang digunakan.

Untuk penelitian kualitatif seperti penelitian tindakan kelas, etnografi, fenomenologi, studi kasus, dan lain-lain, perlu ditambahkan peneliti sebagai instrumen (human instrument), subjek penelitian, informan yang ikut membantu beserta cara-cara menggali data-data penelitian, lokasi dan lama penelitian serta uraian mengenai pengecekan keabsahan hasil penelitian.

Sebaiknya dihindari pengorganisasian penulisan ke dalam “anak sub-judul” pada bagian ini. Namun, jika tidak bisa dihindari, cara penulisannya dapat dilihat pada bagian “Hasil dan Pembahasan”.

Tabel dan Gambar semuanya rata tengah sebagaimana ditunjukkan pada Tabel \ref{tab:tabel_1} dan Gambar \ref{fig:my_gambar1}. Untuk gambar yang lebih dari satu jika memungkinkan dapat diatur sebaris sebagaimana ditunjukkan pada Gambar \ref{fig:my_gambar2}.

\begin{table}[H]
\centering
\fontsize{8pt}{10pt}\selectfont
\caption{Contoh Tabel}
\label{tab:tabel_1}
\begin{tabular}{ccr}
\hline
Kelas & Maksimum & Minimum \\
\hline
x & 10 & 8.6 \\
y & 15 & 12.4 \\
z & 20 & 15.3 \\
\hline
\end{tabular}
\end{table}

\begin{figure}[H]
\centering
\includegraphics{samplepicture}
%% make sure to add \vspace{.7em} before figure's caption
\vspace{.7em}
\caption{Contoh Gambar}
\label{fig:my_gambar1}
\end{figure}

\begin{figure}[H]
\centering
  \subfloat[]{
	\begin{minipage}[c][1.2\width]{0.3\textwidth}
	   \centering
	   \includegraphics[scale=0.33]{figure1}
	\end{minipage}}
 \hspace{2.5cm}	
  \subfloat[]{
	\begin{minipage}[c][1.2\width]{0.3\textwidth}
	   \centering
	   \includegraphics[scale=0.33]{figure1}
	\end{minipage}}
	%% make sure to add \vspace{.7em} before figure's caption
	\vspace{.7em}
\caption{Contoh dua gambar dalam satu baris, \linebreak (a) gambar 1, (b) gambar 2}
\label{fig:my_gambar2}
\end{figure}


\section{HASIL DAN PEMBAHASAN}
\label{hasil}
Bagian ini merupakan bagian utama artikel hasil penelitian dan biasanya merupakan bagian terpanjang dari suatu artikel. Hasil penelitian yang disajikan dalam bagian ini adalah interpretasi hasil analisis. Proses analisis data seperti perhitungan statistik dan proses pengujian hipotesis tidak perlu disajikan. Hanya hasil analisis dan hasil pengujian hipotesis saja yang perlu dilaporkan. Tabel dan grafik dapat digunakan untuk memperjelas penyajian hasil penelitian secara verbal. Tabel dan grafik harus diberi komentar atau dibahas.
Untuk penelitian kualitatif, bagian hasil memuat bagian-bagian rinci dalam bentuk sub topik-sub topik yang berkaitan langsung dengan fokus penelitian dan kategori-kategori.

Pembahasan dalam artikel bertujuan untuk: (1) menjawab rumusan masalah dan (atau) pertanyaan penelitian; (2) menunjukkan bagaimana temuan-temuan itu diperoleh; (3) menginterpretasi/menafsirkan temuan; (4) mengaitkan hasil temuan penelitian dengan struktur pengetahuan yang telah ada; dan (5) memunculkan teori-teori baru atau modifikasi teori yang telah ada.

Dalam menjawab rumusan masalah dan pertanyaan-pertanyaan penelitian, hasil penelitian harus disimpulkan secara eksplisit. Penafsiran terhadap temuan dilakukan dengan menggunakan logika dan teori-teori yang ada. Temuan berupa kenyataan di lapangan diintegrasikan/ dikaitkan dengan hasil-hasil penelitian sebelumnya atau dengan teori yang sudah ada. Untuk keperluan ini harus ada Pustaka. Dalam memunculkan teori-teori baru, teori-teori lama bisa dikonfirmasi atau ditolak, sebagian mungkin perlu memodifikasi teori dari teori lama.

Dalam suatu artikel, kadang-kadang tidak bisa dihindari pengorganisasian penulisan hasil penelitian ke dalam “anak subjudul”. Berikut ini adalah cara menuliskan format pengorganisasian tersebut, yang di dalamnya menunjukkan cara penulisan hal-hal khusus yang tidak dapat dipisahkan dari sebuah artikel.


\subsection{Sub judul 1}
Anda seharusnya menuliskan persamaan dalam Microsoft Equation atau font Symbol. Jika terdapat beberapa persamaan, beri nomor persamaan. Nomor persamaan seharusnya berurutan.

\begin{equation}
E_v - E = \frac{\hbar}{2.m}(k_x^2 + k_y^2)
\end{equation}
%% make sure to add \vspace{.005em} after the end of equation
\vspace{.005em}

\subsection{Sub judul 2}
Singkatan yang sudah umum seperti seperti IEEE, SI, MKS, CGS, sc, dc, and rms tidak perlu diberi keterangan kepanjangannya. Akan tetapi, akronim yang tidak terlalu dikenal atau akronim bikinan penulis perlu diberi keterangan kepanjangannya. Sebagai contoh: Model pembelajaran MiKiR (Multimedia interaktif, Kolaboratif, dan Reflektif) dapat digunakan untuk melatihkan penguasaan keterampilan pemecahan masalah. Jangan gunakan singkatan atau akronim pada judul artikel, kecuali tidak bisa dihindari.

\subsection{Sub judul 3}
Daftar Pustaka merupakan daftar karya tulis yang dibaca penulis dalam mempersiapkan artikelnya dan kemudian digunakan sebagai acuan. Penulisan Daftar Pustaka mengikuti format APA \textit{(American Psychological Association)}. Penulisan Pustaka sebaiknya menggunakan aplikasi Reference Manager seperti Mendeley, Zotero, End Note atau lainnya.

\subsubsection {Sub-sub judul 1}
Jika pada artikel melibatkan penulisan satuan, maka penulisan satuan di dalam artikel memperhatikan aturan sebagai-berikut:
\begin{enumerate}[label=\alph*]
    \item Gunakan SI (MKS) atau CGS sebagai satuan utama, dengan satuan sistem SI lebih diharapkan.
    \item Hindari penggabungan satuan SI dan CGS, karena dapat menimbulkan kerancuan, karena dimensi persamaan bisa menjadi tidak setara.
    \item Jangan mencampur singkatan satuan dengan satuan lengkap. Misalnya, gunakan satuan “\(Wb/m^{2}\)” atau “weber per meter persegi”, jangan “\(weber/m^{2}\)”.
\end{enumerate}

\subsubsection {Sub-sub judul 2}
yy

\section{SIMPULAN}
\label{simpulan}
Simpulan menyajikan ringkasan dari uraian mengenai hasil dan pembahasan, mengacu pada tujuan penelitian. Berdasarkan kedua hal tersebut dikembangkan pokok-pokok pikiran baru yang merupakan esensi dari temuan penelitian.

\section*{UCAPAN TERIMA KASIH (OPSIONAL)}
\label{acknowledgement}
Ucapan teria kasih sifatnya tambahan (opsional).

%% The Appendices part is started with the command \appendix;
%% appendix sections are then done as normal sections
%% \appendix

%% \section{}
%% \label{}

%% References
%%
%% Following citation commands can be used in the body text:
%% Usage of \cite is as follows:
%%   \cite{key}         ==>>  [#]
%%   \cite[chap. 2]{key} ==>> [#, chap. 2]
%%

%% References with BibTeX database:

\printbibliography[title={DAFTAR PUSTAKA}]
%\bibliographystyle{apa7}
%\bibliography{Ijerearchi}
%\bibliography{<your-bib-database>}

%% Authors are advised to use a BibTeX database file for their reference list.
%% The provided style IEEEtran.bst formats references is generally used.

%% For references without a BibTeX database:

\end{document}

%%
%% End of file `iaesarticle.tex'. 